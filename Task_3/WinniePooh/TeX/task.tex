\documentclass[14pt,a4paper]{article} 
\usepackage[margin=1in]{geometry} 
\usepackage[utf8]{inputenc} 
\usepackage[english, russian]{babel} 
\usepackage{amsmath} 
\usepackage{amsfonts}
\usepackage{tcolorbox} 
\usepackage{amssymb} 
\usepackage{amsthm} 
\usepackage{lastpage} 
\usepackage{fancyhdr} 
\usepackage{accents}
\usepackage{tabto}
\usepackage{multicol}
\usepackage{graphicx}
\usepackage{verbatim}
\usepackage{mathdots}
\usepackage{pstricks,pst-node}
\usepackage{alltt}

\parindent = 3em
\parskip = 0pt

\graphicspath{{picks/}}
\DeclareGraphicsExtensions{.png}


\begin{document}
    \begin{titlepage}
        \newpage
         \begin{center}

            Домашнее задание № 3 по курсу Архитектура вычислительных систем.\\
            Практические приемы построения многопоточных приложений\\

            Выполнил : Судаков Дмитрий БПИ 196

            Вариант 22
         \end{center}

    \end{titlepage}


    \section{Задача}
    \par{
        Первая задача о Винни-Пухе, или неправильные пчелы.
        Неправильные пчелы, подсчитав в конце месяца убытки от наличия в лесу
        Винни-Пуха, решили разыскать его и наказать в назидание всем другим
        любителям сладкого. Для поисков медведя они поделили лес на участки,
        каждый из которых прочесывает одна стая неправильных пчел. В случае
        нахождения медведя на своем участке стая проводит показательное
        наказание и возвращается в улей. Если участок прочесан, а Винни-Пух на
        нем не обнаружен, стая также возвращается в улей. Требуется создать
        многопоточное приложение, моделирующее действия пчел. При решении
        использовать парадигму портфеля задач.

        Задача была дополнена : 

        1) Лес задаётся матрицей целых чисел.
        2) Каждый пчелиный отряд, как только освободится (изначально все свободны), пробегается по строке матрицы, и ищет в каждой клетке Винни Пуха.
        3) Винни Пух живёт в той клетке, которая имеет ровно 7 простых делителей.
        4) Как только пчелиный отряд находит Винни Пуха, они производят его наказание путём последовательного вычитания из переменной -1
    }\\
    
    \section{Описание алгоритма}
    \par{
        В цкле по всем строкам матрицы ищется свободный поток, если он был найден, то запускается с передачей ему номера строки в матрицы, мьютекса и номера потока, дальше поток (в массиве потоков) помечается как занятый,  а сам поток после обработки помечает себя как свободный.

        В потоке проходим по строке считаем для каждой клетки количество простых делителей и если оно равно 7, то производится наказание Винни Пуха
    }
    \subsection{Формат входных данных}
    При запуске программы в аргументах командной строки записывается название входного и выходного файла, во входном файле на первой строке должно быть 3 числа количество строк и столбцов матрицы и количество потоков (пчелиных отрядов)
    \subsection{Формат выходных данных}
    В выходной файл выводятся сообщения о начале патрулирования строки, окончании патрулирования строки, нахождении Винни Пуха, наказания Винни Пуха

    \section{Источники}
    Пункт 4 из предложенного списка литературы по многопоточности\\
    Грегори Р. Эндрюс. Основы многопоточного, параллельного и распределенного программирования. - М.: Издательский дом "Вильямс", 2003.

    \section{Тестирование}
    Для генерации тестов был написан небольшой скрипт на питоне, который складывает файлы в папку input. В конце файла всегда написано, местоположение Винни Пуха (для проверки).\\
    Далее для проверки работоспособности был создан bat файл, который запускает программу на фсех тестах и склажывает результирующие файлы в папку output

    
\end{document}